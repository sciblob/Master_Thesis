\section{DONT READ BELOW THIS:}

\subsection{Literature}


Solution control

Material Properties

Boundary conditions

\begin{center}
\captionof{table}[List of boundary conditions]{List of boundary conditions} 
\label{tab:title3}
\begin{tabular}{ | P{5cm} | P{5cm}| P{5cm}| } 
\hline
\textbf{Boundary} & \textbf{Velocity field (U)} & \textbf{Pressure field (p)} \\ 
\hline
inlet & 850.000 & 64  \\ 
\hline
outlet & 1.200.000 & 68\\		
\hline
walltop & 1.500.000 & 67\\
\hline
wall/inletwall/outletwall/stopper/turbostop & 1.500.000 & 67\\
\hline
\end{tabular}
\end{center} 


[1]

In water model experiments, tundish flows under the
influence of flow modifying devices are often studied
with the help of residence time distribution spectra
instead of more complex optical flow measurements. 3–5
Residence time distribution is linked to a simplified
picture of a vessel flow, which is assumed to consist of
different flow regions (tanks) combined in-line or in
parallel. Characteristic features of the residence time
distribution are correlated with these regions, which are
attributed to plug flow, closed recirculating (dead) and
back-mixing. 1

Lagrangian equations:

Besides the classical interpretation of the residence time
distribution spectra, as given in the last subsection, a
more detailed inspection of the data from the tracer can
be done. The time dependent information from an
ensemble of tracer particles is ensemble-averaged at
different times for this purpose. Such an analysis can
help to shed light on the fluid flow history in the tundish.
Figures 6 and 7 show typical results from such evalua-
tions.

[2]

water model; opetimized fluid flow

take some points from introduction

comclusion: slag NMI, turbulence inhibitor turbostop, 

[3]

Thermal buoyancy analysis. Diff equations.
No thermal boy at inlet. Its at far wall regions.
For the
trajectory calculation of inclusions, the Stochastic model
yields more accurate inclusion motion than the non-
Stochastic model. The average residence time of
inclusions decreases with increasing size. The thermal
buoyancy favors inclusions removal especially the small
inclusions. Using solute transport like the dye injection in
water model and copper addition in the real steel tundish
cannot accurately study the motion of the inclusions. In the
simulation, more than 68 inclusions bigger than 10$\mu$m
are removed to the top, and less than 32 enters the mold
which is larger than the industrial measurement.

Though some researchers investigated the steel cleanliness
in the tundish through industrial trials 1-6) , the operation
conditions in steel plants, such as the high temperature, the
visual opacity of the molten steel, and the massive size of
industrial tundishes, hint serious problems for any direct
and elaborate industrial experimental investigations of the
fluid flow phenomena in tundishes. On the other hand,
though the kinematic viscosity of steel is almost equal to
that of water, the fluid flow study through water models 5,
7-20)
also need to be further proved before using to
industrial designs due to the following unclear issues:

Consequently, the numerical simulation becomes a
reasonable alternative to investigate the metal flow in the
tundish and to design the tundish 21-25) . The numerical
simulations of the hydrodynamic phenomena in tundishes
include the single phase turbulent fluid flow, multiphase
fluid flow if gas injected from ladle shroud or tundish soft-
bubbling, residence time distribution, inclusion growth,
motion and removal, mixing and grade transition, thermal
energy transport, or vortexing formation at the start and
the end of casting. Flow optimization in the tundish can be
achieved through the tundish shape, and flow control
devices such as turbulence inhibitors, impact pads, baffles,
weirs and dams. Each tundish is designed in a way as to
realize an optimal flow and therefore higher cleanliness of
the steel by providing 1high average residence time, 2
small severe turbulence, dead and short-circuit volumes,
3 large volume of laminar flow region, 4 forced
coagulation in suitable turbulent zones and floating of
inclusions, assimilated by cover slag, 5avoiding “open
(red) eye” creating uncovered surface of molten steel
against air absorption.

In the tundish, the
inclusion sources are: deoxidation products, ladle lining
erosion product and entrainment of ladle slag (including reoxidation products by SiO 2 , FeO and MnO in slag)
carried over from the ladle; entrainment of the tundish slag
by the excessive top surface level fluctuation especially at
the inlet zone; reoxidation by air in tundish; precipitation
of inclusions at lower temperature, such as TiO 2
inclusions; erosion of the tundish lining. On the other hand, inclusions can be removed by
following mechanisms: buoyancy rising and absorption to
the top slag; fluid flow transport; argon gas bubble
flotation; inclusion growth by collision and Ostwald-
Ripening and floation; inclusion absorption to lining
refractories.
The final inclusion destination includes the top slag, the
lining (safe removal) and mold (possible defects in slab if
not be removed in the mold).
The purpose of fluid flow optimization in the tundish is to
achieve the best flow pattern to remove inclusions from
the molten steel. Hence it is more important to predict the
inclusion motion in the tundish than the fluid flow
simulation itself.

[6]

Take image. 

Slag filteration. However the results obtained with the UDF function
shows better agreement with experimental investigations performed
with water models and at industrial conditions, which show that
inclusions in a range up to 10-20 microm are almost not separated in
the tundish and flowing out to the mould.

Modern equipment for continuous casting of steel are
designed to provide the best possible casting operating conditions.
In practice this means that certain grade of steel is cast in the
longest possible sequences, and change of steel grades are
planned together with the exchange of tundish. Therefore, the
tundish plays an important role in the continuous casting of steel.
It performs several important functions. It provides uniform
distribution of steel to the individual strands and maintains the
desired casting speed. It keeps the temperature on the same level
and protects it from excessive cooling. Current tendency is the
full use of the tundish for the disposal of non-metallic inclusions.
Improving the casting conditions, one can enhance the processes
of separation and flotation of inclusions. The most common way
of using it for this purpose, is to put inside the tundish one of the
flow control devices (FCD) [5-7]. 

Own boundary condition provides critical velocity of the
fluid. This critical velocity decides if particle stays at the surface
or is reflected and re-enter the fluid. The velocity of the fluid in
each cell at the boundary (where particle reach the surface)
is compared with the critical velocity. In case if the flow velocity
is lower than the critical velocity the particle is trapped, otherwise
it is reflected.

[8]

Continous casting process explained. background of CC.

The continuous casting process is not a very old manufacturing process, but in the last
four decades, it became the most common process for producing most basic metals.
One of the particular important components in this system is the tundish.
Traditionally, the tundish acted as a reservoir between the ladle and the mould but
more recently it has been used a grade separator, an inclusion removal device and a
metallurgical reactor.
A continuous drive to understand the molten metal flow
patterns inside the tundish has led to many research papers being published in the
modelling of the flow patterns, through either water modelling or numerical
modelling. Included in these papers, researchers changed the design of the tundish by
adding tundish furniture or changing the dimensions of the tundish. They showed that
these changes can improve the flow patterns that ultimately improve the performance
of the tundish. These changes were however based on trial-and-error experimentation
or relied on the experience of the modeller to propose the design changes. This type
of design methodology can lead to non-optimal solutions.

[9]

Each tundish is designed in a way as to provide an
optimal flow and therefore higher cleanliness of cast steel.
Tundish inside configuration should provide [2]:
as high a value of average retention time as possible,
minimum turbulence and dead and short-circuit volumes,
maximum space for laminar flow of molten steel,
forced coagulation and flotation of non-metallic inclu-
sions, assimilated by cover slag,

[10]

There is scientific agreement on tundish perfor-
mance improvement on incorporating a turbulence inhi-
bitor box or turbostop, but also a dearth of commentary on
the principles of its design. Industrial practice often
employs incredible detail and customization into these
devices, in parts due to this lack of perspective.

Typical objectives of tundish are inclusion separation,
thermochemical modification and homogenization, and
reducing exposure of bulk volume to thermal and compo-
sitional shock during grade transitions or ladle changeover,
thus ensuring efficient operation of downstream casting
assembly. In order to ensure fine degree of control over
composition and cleanliness of solidification product, flow
control devices (FCDs) such as weir, dam and pouring box
are frequently inlayed in tundishes to influence bulk flow
characteristics to favor inclusion removal.
Operational agenda and factors like inclusion distribu-
tion and throughput rate heavily influence decisions
regarding flow modification. However, speaking generally,
a good assessment of inclusion removal efficiency, espe-
cially for larger size inclusions (40 microns or greater) [1],
can be found through the analysis of Residence Time
Distribution (RTD), which characterize flow and mixing in
a tundish [1]. In a related study by the current authors, a
good correspondence has been observed between the two
for water model as well as industrial systems [2].
Defined as the time spent by an incoming fluid element
in the tundish volume, the frequency distribution of resi-
dence time can be obtained by injecting tracer at the inlet
(continuously or as a pulse) and tracking it at the outlet [1].
The key parameters which characterize the ‘C-curves’ so
obtained are, namely the minimum breakthrough time, t min
(normalized as h min ), when the tracer concentration is first
detected at the outlet; time to attain peak tracer concen-
tration, t peak (or h peak ); and t av (or h av ), the mean residence
time for the distribution

tapered turbostop better than straight one.

[11]

increment of residence time. 

take image of turbostop.

(4) The effect of the round turbulence inhibitor on the flow
characteristic of liquid steel in the impact zone is very obvious.
It can improve the mixing capacity of liquid steel, promote the
collision, growth and removal of nonmetallic inclusions and the
homogenization of liquid steel temperature and composition,
and prevent the erosion of high velocity liquid steel stream to
the tundish lining in the impact zone.
(5) The combination of the optimal baffle and round
turbulence inhibitor can improve the flow field and temperature
field in the whole tundish effectively. Especially, flow
characteristic in the farther strand of the tundish is improved and
the difference in flow characteristics between multiple strands
is also smaller. It is worth restating that when optimizing the
whole performance of multi-strand tundish, not only the flow
characteristic of each strand but also the difference of flow
characteristics between multiple strands should be considered

[13]

Isothermal and non-isothermal flows. An inspection of
the isothermal and non-isothermal flow shows that there
are only small differences between these cases for all in-
vestigated tundishes. Therefore, thermal natural convection
does not playa significant role in these tundish flows. In
detail, the temperature difference from the inlet nozzle to
the outlet nozzle is about 8 - 10K for all configurations.
This result differs significantly from the findings of other
authors, i.e. Joo and Guthrie [15]. They have reported a
temperature difference of about 16 K for a twin-slab tun-
dish under similar conditions. They also report a noticeable
influence of the thermal natural convection in the non-
isothermal flows. It must be noticed that their model is
based on an energy transport equation with a fixed turbu-
lent Prandtl number.

[14]

take image.

[16]

The tundish plays a key role in the process of continuous steel casting (CSC), as it ensures a stable
inflow of steel to the mould. Slabs formed in this process should exhibit the required quality, which is
determined by tundish operation, among other factors. The quality of a slabs is determined primarily
by the steel overheating temperature and the level of impurities in the form of non-metallic inclusions
(NMIs) and gases. The impurity level prescribed for a given steel grade is achieved during refining in the
ladle furnace. While the tundish should either maintain or reduce the impurity level achieved in the ladle
furnace. Therefore, particular attention is paid to the behavior of the tundish powder and the ceramic
tundish working layer, which may cause the formation of additional non-metallic inclusions. Advanced
tundish stations are equipped with plasma reheating or circulatory degassing installations [1,2]. While
as standard, tundishes are furnished with flow control devices (FCD), NMI flotation intensifying or
chemical and thermal homogenization enhancing devices. Classic flow control devices include dams,
weirs, subflux turbulence controllers (STC), or argon curtains [3–14]. Often, tundishes are equipped
with ceramic filters, or special chambers to induce rotary motion [15,16]. To modify the liquid steel
motion in a contactless manner, electromagnetic stirring is used [17–19]. One of the most common
FCDs is a STC installed in the pouring zone. The main purpose of the STC is to suppress the turbulent
flow beyond the tundish pouring zone. The STC is a device in the form of a flanged box. The internal
working part of the STC is formed by walls that often have a varying angle of inclination relative to
the STC bottom. As standard, an STC is installed in the axis of the feeding stream flowing out of the
steelmaking ladle.

[17]

tundish and RTD imaage; water model image.

Tundish is a flow reactor, which is equipped with dif-
ferent flow control devices such as: baffles, notches, dams,
impact pads, gaseous-permeable membranes or filters [1-4].
Today the most commonly used equipment of tundish seem
to be turbulence inhibitors, also called subflux controllers of
turbulence. Such devices are positioned in the axis of steel
flux flowing into tundish. Turbulence inhibitors have different
shapes; however their size and configuration of external walls
have the essential meaning on the shaping of steel flow.
The devices mentioned above definitely influence flowing
conditions in the working zone of a tundish. As a consequence
they also influence the quality of obtained continuous casting
ingot. In general the type, size and fitting place of the flow
control device is chosen individually for a given tundish.

Applying turbulence inhibitor limits the dynamics of steel
flux coming into tundish, then the flow of steel becomes
more uniform (lower profile of velocity). It gives better
control over the steel flow.
Turbulence inhibitors have beneficial influence on dura-
bility of refractory lining in near-inlet zone, therefore the
consumption of such lining during the process of filling
the tundish with liquid steel is lower.
Comparing with the tundish without impact pads, appli-
cation of the turbulence inhibitor makes the liquid steel
better mixed. This improvement is based on increasing the
dynamics of liquid steel mixing, so then the zone of inten-
sive mixing is also bigger. Such situation favours floating
the nonmetallic inclusions to the slag and their absorp-
tion. This concerns all analyzed variants of turbulence
inhibitors.
Applying turbulence inhibitors decrease the percentage
participation of dead zones (see Figure 8). However, it is not possible to eliminate them. Areas most susceptible
to this flow disturbance are the ones near back wall of the
tundish in a near-slag zone.
Turbulence inhibitors influence shaping the zone of inten-
sive mixing, but forming the zone of dispersed plug in
a channel part of the tundish is rather difficult. This is
caused by the movement of liquid downwards after be-
ing reflected from the slag. However, the time the tracer
reaches particular outlets is rather similar. So, it can be
assumed that the course of casting process will be correct.
The shape of turbulence inhibitor (square – variant A or
round – variant D) does not influence the kinetics of steel
mixing and also the ratio of percentage participation of
particular flow volume in the tundish (Table 1, Fig. 8).
Formation of inner space of turbulence inhibitor influences
the flow and kinetics of steel mixing and also the ratio of
percentage participation of particular flow volume in the
tundish (Table 1, Fig. 8).
According to the rule concerning the correctly modernized
tundish, which says that to improve the refining capacity
of a tundish the dead zone should be minimized and the
dispersed plug flow volume should be increased even at
the cost of well mixed flow volume, the most effective
seems to be geometry of turbulence inhibitor variant D.

[18]

first 22 pages.

[20]

continous casting process.

[21]

first 14 pages. 

[23]

The aperture ratio f3 of the
turbo-stopper, defined as the ratio of shroud diameter to
turbo-stopper outlet diameter, should be small and the
opening angle a should be large. This causes a high up-
ward stream out of the turbo-stopper and an intensive
shear layer formation.

\pagebreak

\subsection{Implementation}

\[ \frac{\partial u}{\partial t}
   = h^2 \left( \frac{\partial^2 u}{\partial x^2}
      + \frac{\partial^2 u}{\partial y^2}
      + \frac{\partial^2 u}{\partial z^2} \right) \]

\[
\dv{Q}{t} = \dv{s}{t}  \quad
\dv[n]{Q}{t} = \dv[n]{s}{t}  \quad
\pdv{Q}{t} = \pdv{s}{t}  \quad
\pdv[n]{Q}{t} = \pdv[n]{s}{t}  \quad
\pdv{Q}{x}{t} = \pdv{s}{x}{t}  \quad
\]
\[
\fdv{F}{g}
\]
